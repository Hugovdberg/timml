%% Generated by Sphinx.
\def\sphinxdocclass{report}
\documentclass[letterpaper,10pt,english]{sphinxmanual}
\ifdefined\pdfpxdimen
   \let\sphinxpxdimen\pdfpxdimen\else\newdimen\sphinxpxdimen
\fi \sphinxpxdimen=.75bp\relax

\usepackage[utf8]{inputenc}
\ifdefined\DeclareUnicodeCharacter
 \ifdefined\DeclareUnicodeCharacterAsOptional\else
  \DeclareUnicodeCharacter{00A0}{\nobreakspace}
\fi\fi
\usepackage{cmap}
\usepackage[T1]{fontenc}
\usepackage{amsmath,amssymb,amstext}
\usepackage{babel}
\usepackage{times}
\usepackage[Bjarne]{fncychap}
\usepackage[dontkeepoldnames]{sphinx}

\usepackage{geometry}

% Include hyperref last.
\usepackage{hyperref}
% Fix anchor placement for figures with captions.
\usepackage{hypcap}% it must be loaded after hyperref.
% Set up styles of URL: it should be placed after hyperref.
\urlstyle{same}

\addto\captionsenglish{\renewcommand{\figurename}{Fig.}}
\addto\captionsenglish{\renewcommand{\tablename}{Table}}
\addto\captionsenglish{\renewcommand{\literalblockname}{Listing}}

\addto\extrasenglish{\def\pageautorefname{page}}

\setcounter{tocdepth}{2}



\title{TimML Documentation}
\date{Oct 06, 2017}
\release{5.0.0}
\author{M. Bakker}
\newcommand{\sphinxlogo}{\vbox{}}
\renewcommand{\releasename}{Release}
\makeindex

\begin{document}

\maketitle
\sphinxtableofcontents
\phantomsection\label{\detokenize{index::doc}}


TimML is a computer program for the modeling of steady-state multi-layer flow with analytic elements
TimML may be applied to an arbitrary number of layers and arbitrary sequence of aquifers and leaky layers.
The Dupuit approximation is applied to aquifer layers, while flow in leaky layers is approximated as vertical.
The head, flow, and leakage between aquifer layers may be computed analytically at any point in the aquifer system.
The design of TimML is object-oriented and has been kept simple and flexible.
New analytic elements may be added to the code without making any changes in the existing part of the code.
TimML is coded in Python. Behind the scenes, use is made of FORTRAN extensions to improve performance.


\chapter{Installation}
\label{\detokenize{index:installation}}\label{\detokenize{index:timml-this-documentation-is-under-construction}}
TimML is written for Python 3.
\begin{quote}

pip install instructions will follwo soon.
\end{quote}


\chapter{Main Approximations}
\label{\detokenize{index:main-approximations}}

\chapter{List of available elements}
\label{\detokenize{index:list-of-available-elements}}\begin{itemize}
\item {} 
Well
\begin{itemize}
\item {} 
Discharge-specified well

\item {} 
Head-specified well

\item {} 
Multi-aquifer well. Well is screened in multiple layers and only total discharge is specified.

\end{itemize}

\item {} 
Line-sink
\begin{itemize}
\item {} 
Head-specified line-sink

\item {} 
String of head-specified line-sinks

\end{itemize}

\end{itemize}


\section{Models}
\label{\detokenize{models/modelindex::doc}}\label{\detokenize{models/modelindex:models}}

\subsection{Multi-Aquifer Model}
\label{\detokenize{models/modelmaq::doc}}\label{\detokenize{models/modelmaq:multi-aquifer-model}}\index{ModelMaq (class in timml.model)}

\begin{fulllineitems}
\phantomsection\label{\detokenize{models/modelmaq:timml.model.ModelMaq}}\pysiglinewithargsret{\sphinxstrong{class }\sphinxcode{timml.model.}\sphinxbfcode{ModelMaq}}{\emph{kaq=1, z={[}1, 0{]}, c={[}{]}, npor=0.3, top=’conf’, hstar=None}}{}
ModelMaq Class to create a multi-aquifer model object
\begin{quote}\begin{description}
\item[{Parameters}] \leavevmode\begin{itemize}
\item {} 
\sphinxstyleliteralstrong{kaq} (\sphinxstyleliteralemphasis{float}\sphinxstyleliteralemphasis{, }\sphinxstyleliteralemphasis{array}\sphinxstyleliteralemphasis{ or }\sphinxstyleliteralemphasis{list}) \textendash{} hydraulic conductivity of each aquifer from the top down
if float, hydraulic conductivity is the same in all aquifers

\item {} 
\sphinxstyleliteralstrong{z} (\sphinxstyleliteralemphasis{array}\sphinxstyleliteralemphasis{ or }\sphinxstyleliteralemphasis{list}) \textendash{} elevation tops and bottoms of the aquifers from the top down
leaky layers may have zero thickness
if top=’conf’: length is 2 * number of aquifers
if top=’semi’: length is 2 * number of aquifers + 1 as top
of leaky layer on top of systems needs to be specified

\item {} 
\sphinxstyleliteralstrong{c} (\sphinxstyleliteralemphasis{float}\sphinxstyleliteralemphasis{, }\sphinxstyleliteralemphasis{array}\sphinxstyleliteralemphasis{ or }\sphinxstyleliteralemphasis{list}) \textendash{} resistance of leaky layers from the top down
if float, resistance is the same for all leaky layers
if top=’conf’: length is number of aquifers - 1
if top=’semi’: length is number of aquifers

\item {} 
\sphinxstyleliteralstrong{npor} (\sphinxstyleliteralemphasis{float}\sphinxstyleliteralemphasis{, }\sphinxstyleliteralemphasis{array}\sphinxstyleliteralemphasis{ or }\sphinxstyleliteralemphasis{list}) \textendash{} porosity of all aquifers and leaky layers from the top down
if float, porosity is the same for all layers
if top=’conf’: length is 2 * number of aquifers - 1
if top=’semi’: length is 2 * number of aquifers

\item {} 
\sphinxstyleliteralstrong{top} (\sphinxstyleliteralemphasis{string}\sphinxstyleliteralemphasis{, }\sphinxstyleliteralemphasis{'conf'}\sphinxstyleliteralemphasis{ or }\sphinxstyleliteralemphasis{'semi'}\sphinxstyleliteralemphasis{ (}\sphinxstyleliteralemphasis{default is 'conf'}\sphinxstyleliteralemphasis{)}) \textendash{} indicating whether the top is confined (‘conf’) or
semi-confined (‘semi’)

\item {} 
\sphinxstyleliteralstrong{hstar} (\sphinxstyleliteralemphasis{float}\sphinxstyleliteralemphasis{ or }\sphinxstyleliteralemphasis{None}\sphinxstyleliteralemphasis{ (}\sphinxstyleliteralemphasis{default is None}\sphinxstyleliteralemphasis{)}) \textendash{} head value above semi-confining top, only read if top=’semi’

\end{itemize}

\end{description}\end{quote}
\paragraph{Examples}

\begin{sphinxVerbatim}[commandchars=\\\{\}]
\PYG{g+gp}{\PYGZgt{}\PYGZgt{}\PYGZgt{} }\PYG{n}{ml} \PYG{o}{=} \PYG{n}{ModelMaq}\PYG{p}{(}\PYG{n}{kaq}\PYG{o}{=}\PYG{p}{[}\PYG{l+m+mi}{10}\PYG{p}{,} \PYG{l+m+mi}{20}\PYG{p}{]}\PYG{p}{,} \PYG{n}{z}\PYG{o}{=}\PYG{p}{[}\PYG{l+m+mi}{20}\PYG{p}{,} \PYG{l+m+mi}{12}\PYG{p}{,} \PYG{l+m+mi}{10}\PYG{p}{,} \PYG{l+m+mi}{0}\PYG{p}{]}\PYG{p}{,} \PYG{n}{c}\PYG{o}{=}\PYG{l+m+mi}{1000}\PYG{p}{)}
\end{sphinxVerbatim}
\index{disvec() (timml.model.ModelMaq method)}

\begin{fulllineitems}
\phantomsection\label{\detokenize{models/modelmaq:timml.model.ModelMaq.disvec}}\pysiglinewithargsret{\sphinxbfcode{disvec}}{\emph{x}, \emph{y}, \emph{aq=None}}{}
Discharge vector at \sphinxtitleref{x}, \sphinxtitleref{y}
\begin{quote}\begin{description}
\item[{Returns}] \leavevmode
\sphinxstylestrong{qxqy} \textendash{} first row is Qx in each aquifer layer, second row is Qy

\item[{Return type}] \leavevmode
array size (2, naq)

\end{description}\end{quote}

\end{fulllineitems}

\index{head() (timml.model.ModelMaq method)}

\begin{fulllineitems}
\phantomsection\label{\detokenize{models/modelmaq:timml.model.ModelMaq.head}}\pysiglinewithargsret{\sphinxbfcode{head}}{\emph{x}, \emph{y}, \emph{layers=None}, \emph{aq=None}}{}
Head at \sphinxtitleref{x}, \sphinxtitleref{y}
\begin{quote}\begin{description}
\item[{Returns}] \leavevmode
\sphinxstylestrong{h} \textendash{} head in all \sphinxtitleref{layers} (if not \sphinxtitleref{None}), or all layers of aquifer (otherwise)

\item[{Return type}] \leavevmode
array length \sphinxtitleref{naq} or \sphinxtitleref{len(layers)}

\end{description}\end{quote}

\end{fulllineitems}

\index{headalongline() (timml.model.ModelMaq method)}

\begin{fulllineitems}
\phantomsection\label{\detokenize{models/modelmaq:timml.model.ModelMaq.headalongline}}\pysiglinewithargsret{\sphinxbfcode{headalongline}}{\emph{x}, \emph{y}, \emph{layers=None}}{}
Returns head{[}Nlayers,len(x){]}
Assumes same number of layers for each x and y
layers may be None or list of layers for which head is computed

\end{fulllineitems}

\index{headgrid() (timml.model.ModelMaq method)}

\begin{fulllineitems}
\phantomsection\label{\detokenize{models/modelmaq:timml.model.ModelMaq.headgrid}}\pysiglinewithargsret{\sphinxbfcode{headgrid}}{\emph{xg}, \emph{yg}, \emph{layers=None}, \emph{printrow=False}}{}
Returns h{[}Nlayers,ny,nx{]}.
If layers is None, all layers are returned

\end{fulllineitems}

\index{headgrid2() (timml.model.ModelMaq method)}

\begin{fulllineitems}
\phantomsection\label{\detokenize{models/modelmaq:timml.model.ModelMaq.headgrid2}}\pysiglinewithargsret{\sphinxbfcode{headgrid2}}{\emph{x1}, \emph{x2}, \emph{nx}, \emph{y1}, \emph{y2}, \emph{ny}, \emph{layers=None}, \emph{printrow=False}}{}
Returns h{[}Nlayers,ny,nx{]}. If layers is None, all layers are returned

\end{fulllineitems}

\index{remove\_element() (timml.model.ModelMaq method)}

\begin{fulllineitems}
\phantomsection\label{\detokenize{models/modelmaq:timml.model.ModelMaq.remove_element}}\pysiglinewithargsret{\sphinxbfcode{remove\_element}}{\emph{e}}{}
Remove element \sphinxtitleref{e} from model

\end{fulllineitems}

\index{solve() (timml.model.ModelMaq method)}

\begin{fulllineitems}
\phantomsection\label{\detokenize{models/modelmaq:timml.model.ModelMaq.solve}}\pysiglinewithargsret{\sphinxbfcode{solve}}{\emph{printmat=0}, \emph{sendback=0}, \emph{silent=False}}{}
Compute solution

\end{fulllineitems}


\end{fulllineitems}



\subsection{Three-dimensional flow model}
\label{\detokenize{models/model3d::doc}}\label{\detokenize{models/model3d:three-dimensional-flow-model}}
Similar to LPF in MODFLOW
\index{Model3D (class in timml.model)}

\begin{fulllineitems}
\phantomsection\label{\detokenize{models/model3d:timml.model.Model3D}}\pysiglinewithargsret{\sphinxstrong{class }\sphinxcode{timml.model.}\sphinxbfcode{Model3D}}{\emph{kaq=1, z={[}1, 0{]}, kzoverkh=1, npor=0.3, top=’conf’, topres=0, topthick=0, hstar=0}}{}
Model3D Class to create a multi-layer model object consisting of
many aquifer layers. The resistance between the layers is computed
from the vertical hydraulic conductivity of the layers.
\begin{quote}\begin{description}
\item[{Parameters}] \leavevmode\begin{itemize}
\item {} 
\sphinxstyleliteralstrong{kaq} (\sphinxstyleliteralemphasis{float}\sphinxstyleliteralemphasis{, }\sphinxstyleliteralemphasis{array}\sphinxstyleliteralemphasis{ or }\sphinxstyleliteralemphasis{list}) \textendash{} hydraulic conductivity of each layer from the top down
if float, hydraulic conductivity is the same in all aquifers

\item {} 
\sphinxstyleliteralstrong{z} (\sphinxstyleliteralemphasis{array}\sphinxstyleliteralemphasis{ or }\sphinxstyleliteralemphasis{list}) \textendash{} elevation of top of system followed by bottoms of all layers
from the top down
bottom of layer is automatically equal to top of layer below it
if top=’conf’: length is number of layers + 1
if top=’semi’: length is number of layers + 2 as top
of leaky layer on top of systems needs to be specified

\item {} 
\sphinxstyleliteralstrong{kzoverkh} (\sphinxstyleliteralemphasis{float}) \textendash{} vertical anisotropy ratio vertical k divided by horizontal k
if float, value is the same for all layers
length is number of layers

\item {} 
\sphinxstyleliteralstrong{npor} (\sphinxstyleliteralemphasis{float}\sphinxstyleliteralemphasis{, }\sphinxstyleliteralemphasis{array}\sphinxstyleliteralemphasis{ or }\sphinxstyleliteralemphasis{list}) \textendash{} porosity of all aquifer layers from the top down
if float, porosity is the same for all layers
if top=’conf’: length is number of layers
if top=’semi’: length is number of layers + 1

\item {} 
\sphinxstyleliteralstrong{top} (\sphinxstyleliteralemphasis{string}\sphinxstyleliteralemphasis{, }\sphinxstyleliteralemphasis{'conf'}\sphinxstyleliteralemphasis{ or }\sphinxstyleliteralemphasis{'semi'}\sphinxstyleliteralemphasis{ (}\sphinxstyleliteralemphasis{default is 'conf'}\sphinxstyleliteralemphasis{)}) \textendash{} indicating whether the top is confined (‘conf’) or
semi-confined (‘semi’)

\item {} 
\sphinxstyleliteralstrong{topres} (\sphinxstyleliteralemphasis{float}) \textendash{} resistance of top semi-confining layer, only read if top=’semi’

\item {} 
\sphinxstyleliteralstrong{topthick} (\sphinxstyleliteralemphasis{float}) \textendash{} thickness of top semi-confining layer, only read if top=’semi’

\item {} 
\sphinxstyleliteralstrong{hstar} (\sphinxstyleliteralemphasis{float}\sphinxstyleliteralemphasis{ or }\sphinxstyleliteralemphasis{None}\sphinxstyleliteralemphasis{ (}\sphinxstyleliteralemphasis{default is None}\sphinxstyleliteralemphasis{)}) \textendash{} head value above semi-confining top, only read if top=’semi’

\end{itemize}

\end{description}\end{quote}
\paragraph{Examples}

\begin{sphinxVerbatim}[commandchars=\\\{\}]
\PYG{g+gp}{\PYGZgt{}\PYGZgt{}\PYGZgt{} }\PYG{n}{ml} \PYG{o}{=} \PYG{n}{Model3D}\PYG{p}{(}\PYG{n}{kaq}\PYG{o}{=}\PYG{l+m+mi}{10}\PYG{p}{,} \PYG{n}{z}\PYG{o}{=}\PYG{n}{np}\PYG{o}{.}\PYG{n}{arange}\PYG{p}{(}\PYG{l+m+mi}{20}\PYG{p}{,} \PYG{o}{\PYGZhy{}}\PYG{l+m+mi}{1}\PYG{p}{,} \PYG{o}{\PYGZhy{}}\PYG{l+m+mi}{2}\PYG{p}{)}\PYG{p}{,} \PYG{n}{kzoverkh}\PYG{o}{=}\PYG{l+m+mf}{0.1}\PYG{p}{)}
\end{sphinxVerbatim}
\index{disvec() (timml.model.Model3D method)}

\begin{fulllineitems}
\phantomsection\label{\detokenize{models/model3d:timml.model.Model3D.disvec}}\pysiglinewithargsret{\sphinxbfcode{disvec}}{\emph{x}, \emph{y}, \emph{aq=None}}{}
Discharge vector at \sphinxtitleref{x}, \sphinxtitleref{y}
\begin{quote}\begin{description}
\item[{Returns}] \leavevmode
\sphinxstylestrong{qxqy} \textendash{} first row is Qx in each aquifer layer, second row is Qy

\item[{Return type}] \leavevmode
array size (2, naq)

\end{description}\end{quote}

\end{fulllineitems}

\index{head() (timml.model.Model3D method)}

\begin{fulllineitems}
\phantomsection\label{\detokenize{models/model3d:timml.model.Model3D.head}}\pysiglinewithargsret{\sphinxbfcode{head}}{\emph{x}, \emph{y}, \emph{layers=None}, \emph{aq=None}}{}
Head at \sphinxtitleref{x}, \sphinxtitleref{y}
\begin{quote}\begin{description}
\item[{Returns}] \leavevmode
\sphinxstylestrong{h} \textendash{} head in all \sphinxtitleref{layers} (if not \sphinxtitleref{None}), or all layers of aquifer (otherwise)

\item[{Return type}] \leavevmode
array length \sphinxtitleref{naq} or \sphinxtitleref{len(layers)}

\end{description}\end{quote}

\end{fulllineitems}

\index{headalongline() (timml.model.Model3D method)}

\begin{fulllineitems}
\phantomsection\label{\detokenize{models/model3d:timml.model.Model3D.headalongline}}\pysiglinewithargsret{\sphinxbfcode{headalongline}}{\emph{x}, \emph{y}, \emph{layers=None}}{}
Returns head{[}Nlayers,len(x){]}
Assumes same number of layers for each x and y
layers may be None or list of layers for which head is computed

\end{fulllineitems}

\index{headgrid() (timml.model.Model3D method)}

\begin{fulllineitems}
\phantomsection\label{\detokenize{models/model3d:timml.model.Model3D.headgrid}}\pysiglinewithargsret{\sphinxbfcode{headgrid}}{\emph{xg}, \emph{yg}, \emph{layers=None}, \emph{printrow=False}}{}
Returns h{[}Nlayers,ny,nx{]}.
If layers is None, all layers are returned

\end{fulllineitems}

\index{headgrid2() (timml.model.Model3D method)}

\begin{fulllineitems}
\phantomsection\label{\detokenize{models/model3d:timml.model.Model3D.headgrid2}}\pysiglinewithargsret{\sphinxbfcode{headgrid2}}{\emph{x1}, \emph{x2}, \emph{nx}, \emph{y1}, \emph{y2}, \emph{ny}, \emph{layers=None}, \emph{printrow=False}}{}
Returns h{[}Nlayers,ny,nx{]}. If layers is None, all layers are returned

\end{fulllineitems}

\index{remove\_element() (timml.model.Model3D method)}

\begin{fulllineitems}
\phantomsection\label{\detokenize{models/model3d:timml.model.Model3D.remove_element}}\pysiglinewithargsret{\sphinxbfcode{remove\_element}}{\emph{e}}{}
Remove element \sphinxtitleref{e} from model

\end{fulllineitems}

\index{solve() (timml.model.Model3D method)}

\begin{fulllineitems}
\phantomsection\label{\detokenize{models/model3d:timml.model.Model3D.solve}}\pysiglinewithargsret{\sphinxbfcode{solve}}{\emph{printmat=0}, \emph{sendback=0}, \emph{silent=False}}{}
Compute solution

\end{fulllineitems}


\end{fulllineitems}



\subsection{Model}
\label{\detokenize{models/model::doc}}\label{\detokenize{models/model:model}}\index{Model (class in timml.model)}

\begin{fulllineitems}
\phantomsection\label{\detokenize{models/model:timml.model.Model}}\pysiglinewithargsret{\sphinxstrong{class }\sphinxcode{timml.model.}\sphinxbfcode{Model}}{\emph{kaq}, \emph{c}, \emph{z}, \emph{npor}, \emph{ltype}}{}
Model Class to create a model object consisting of an arbitrary
sequence of aquifer layers and leaky layers.
Use ModelMaq for regular sequence of aquifer and leaky layers.
Use Model3D for multi-layer model of single aquifer
\begin{quote}\begin{description}
\item[{Parameters}] \leavevmode\begin{itemize}
\item {} 
\sphinxstyleliteralstrong{kaq} (\sphinxstyleliteralemphasis{array}) \textendash{} hydraulic conductivity of each aquifer from the top down

\item {} 
\sphinxstyleliteralstrong{z} (\sphinxstyleliteralemphasis{array}) \textendash{} elevation tops and bottoms of all layers
layers may have zero thickness

\item {} 
\sphinxstyleliteralstrong{c} (\sphinxstyleliteralemphasis{array}) \textendash{} resistance between two consecutive aquifers
if ltype{[}0{]}=’a’: length is number of aquifers - 1
if ltype{[}0{]}=’l’: length is number of aquifers

\item {} 
\sphinxstyleliteralstrong{npor} (\sphinxstyleliteralemphasis{array}) \textendash{} porosity of all layers from the top down

\item {} 
\sphinxstyleliteralstrong{ltype} (\sphinxstyleliteralemphasis{array of characters}) \textendash{} array indicating for each layer whether it is
‘a’ aquifer layer
‘l’ leaky layer

\end{itemize}

\end{description}\end{quote}
\paragraph{Examples}

\begin{sphinxVerbatim}[commandchars=\\\{\}]
\PYG{g+gp}{\PYGZgt{}\PYGZgt{}\PYGZgt{} }\PYG{k+kn}{from} \PYG{n+nn}{timml} \PYG{k}{import} \PYG{o}{*}
\PYG{g+gp}{\PYGZgt{}\PYGZgt{}\PYGZgt{} }\PYG{n}{ml} \PYG{o}{=} \PYG{n}{Model}\PYG{p}{(}\PYG{n}{kaq}\PYG{o}{=}\PYG{n}{array}\PYG{p}{(}\PYG{p}{[}\PYG{l+m+mi}{10}\PYG{p}{,} \PYG{l+m+mi}{20}\PYG{p}{,} \PYG{l+m+mi}{10}\PYG{p}{]}\PYG{p}{)}\PYG{p}{,} \PYG{n}{c}\PYG{o}{=}\PYG{n}{array}\PYG{p}{(}\PYG{p}{[}\PYG{l+m+mi}{200}\PYG{p}{,} \PYG{l+m+mi}{2000}\PYG{p}{]}\PYG{p}{)}\PYG{p}{,}              \PYG{n}{z}\PYG{o}{=}\PYG{n}{array}\PYG{p}{(}\PYG{p}{[}\PYG{l+m+mi}{20}\PYG{p}{,} \PYG{l+m+mi}{15}\PYG{p}{,} \PYG{l+m+mi}{10}\PYG{p}{,} \PYG{l+m+mi}{8}\PYG{p}{,} \PYG{l+m+mi}{0}\PYG{p}{]}\PYG{p}{)}\PYG{p}{,} \PYG{n}{npor}\PYG{o}{=}\PYG{l+m+mf}{0.3} \PYG{o}{*} \PYG{n}{ones}\PYG{p}{(}\PYG{l+m+mi}{4}\PYG{p}{)}\PYG{p}{,}              \PYG{n}{ltype}\PYG{o}{=}\PYG{n}{array}\PYG{p}{(}\PYG{p}{[}\PYG{l+s+s1}{\PYGZsq{}}\PYG{l+s+s1}{a}\PYG{l+s+s1}{\PYGZsq{}}\PYG{p}{,} \PYG{l+s+s1}{\PYGZsq{}}\PYG{l+s+s1}{a}\PYG{l+s+s1}{\PYGZsq{}}\PYG{p}{,} \PYG{l+s+s1}{\PYGZsq{}}\PYG{l+s+s1}{l}\PYG{l+s+s1}{\PYGZsq{}}\PYG{p}{,} \PYG{l+s+s1}{\PYGZsq{}}\PYG{l+s+s1}{a}\PYG{l+s+s1}{\PYGZsq{}}\PYG{p}{]}\PYG{p}{)}\PYG{p}{)}
\end{sphinxVerbatim}
\index{disvec() (timml.model.Model method)}

\begin{fulllineitems}
\phantomsection\label{\detokenize{models/model:timml.model.Model.disvec}}\pysiglinewithargsret{\sphinxbfcode{disvec}}{\emph{x}, \emph{y}, \emph{aq=None}}{}
Discharge vector at \sphinxtitleref{x}, \sphinxtitleref{y}
\begin{quote}\begin{description}
\item[{Returns}] \leavevmode
\sphinxstylestrong{qxqy} \textendash{} first row is Qx in each aquifer layer, second row is Qy

\item[{Return type}] \leavevmode
array size (2, naq)

\end{description}\end{quote}

\end{fulllineitems}

\index{head() (timml.model.Model method)}

\begin{fulllineitems}
\phantomsection\label{\detokenize{models/model:timml.model.Model.head}}\pysiglinewithargsret{\sphinxbfcode{head}}{\emph{x}, \emph{y}, \emph{layers=None}, \emph{aq=None}}{}
Head at \sphinxtitleref{x}, \sphinxtitleref{y}
\begin{quote}\begin{description}
\item[{Returns}] \leavevmode
\sphinxstylestrong{h} \textendash{} head in all \sphinxtitleref{layers} (if not \sphinxtitleref{None}), or all layers of aquifer (otherwise)

\item[{Return type}] \leavevmode
array length \sphinxtitleref{naq} or \sphinxtitleref{len(layers)}

\end{description}\end{quote}

\end{fulllineitems}

\index{headalongline() (timml.model.Model method)}

\begin{fulllineitems}
\phantomsection\label{\detokenize{models/model:timml.model.Model.headalongline}}\pysiglinewithargsret{\sphinxbfcode{headalongline}}{\emph{x}, \emph{y}, \emph{layers=None}}{}
Returns head{[}Nlayers,len(x){]}
Assumes same number of layers for each x and y
layers may be None or list of layers for which head is computed

\end{fulllineitems}

\index{headgrid() (timml.model.Model method)}

\begin{fulllineitems}
\phantomsection\label{\detokenize{models/model:timml.model.Model.headgrid}}\pysiglinewithargsret{\sphinxbfcode{headgrid}}{\emph{xg}, \emph{yg}, \emph{layers=None}, \emph{printrow=False}}{}
Returns h{[}Nlayers,ny,nx{]}.
If layers is None, all layers are returned

\end{fulllineitems}

\index{headgrid2() (timml.model.Model method)}

\begin{fulllineitems}
\phantomsection\label{\detokenize{models/model:timml.model.Model.headgrid2}}\pysiglinewithargsret{\sphinxbfcode{headgrid2}}{\emph{x1}, \emph{x2}, \emph{nx}, \emph{y1}, \emph{y2}, \emph{ny}, \emph{layers=None}, \emph{printrow=False}}{}
Returns h{[}Nlayers,ny,nx{]}. If layers is None, all layers are returned

\end{fulllineitems}

\index{remove\_element() (timml.model.Model method)}

\begin{fulllineitems}
\phantomsection\label{\detokenize{models/model:timml.model.Model.remove_element}}\pysiglinewithargsret{\sphinxbfcode{remove\_element}}{\emph{e}}{}
Remove element \sphinxtitleref{e} from model

\end{fulllineitems}

\index{solve() (timml.model.Model method)}

\begin{fulllineitems}
\phantomsection\label{\detokenize{models/model:timml.model.Model.solve}}\pysiglinewithargsret{\sphinxbfcode{solve}}{\emph{printmat=0}, \emph{sendback=0}, \emph{silent=False}}{}
Compute solution

\end{fulllineitems}


\end{fulllineitems}



\section{Inhomogeneities}
\label{\detokenize{inhoms/inhoms::doc}}\label{\detokenize{inhoms/inhoms:inhomogeneities}}

\section{Elements}
\label{\detokenize{aems:elements}}\label{\detokenize{aems::doc}}

\subsection{Wells}
\label{\detokenize{wells/wellindex:wells}}\label{\detokenize{wells/wellindex::doc}}

\subsubsection{Well}
\label{\detokenize{wells/well::doc}}\label{\detokenize{wells/well:well}}\index{Well (class in timml.well)}

\begin{fulllineitems}
\phantomsection\label{\detokenize{wells/well:timml.well.Well}}\pysiglinewithargsret{\sphinxstrong{class }\sphinxcode{timml.well.}\sphinxbfcode{Well}}{\emph{model}, \emph{xw=0}, \emph{yw=0}, \emph{Qw=100.0}, \emph{rw=0.1}, \emph{res=0.0}, \emph{layers=0}, \emph{label=None}}{}
Well Class to create a well with a specified discharge. The well
may be screened in multiple layers. The resistance of the screen may
be specified. The head is computed such that the discharge \(Q_i\)
in layer \(i\) is computed as
\begin{equation*}
\begin{split}Q_i = 2\pi r_w(h_i - h_w)/c\end{split}
\end{equation*}
where \(c\) is the resistance of the well screen and \(h_w\) is
the head inside the well. The total discharge is distributed over the
screens such that \(h_w\) is the same in each screened layer.
\begin{quote}\begin{description}
\item[{Parameters}] \leavevmode\begin{itemize}
\item {} 
\sphinxstyleliteralstrong{model} (\sphinxstyleliteralemphasis{Model object}) \textendash{} model to which the element is added

\item {} 
\sphinxstyleliteralstrong{xw} (\sphinxstyleliteralemphasis{float}) \textendash{} x-coordinate of the well

\item {} 
\sphinxstyleliteralstrong{yw} (\sphinxstyleliteralemphasis{float}) \textendash{} y-coordinate of the well

\item {} 
\sphinxstyleliteralstrong{Qw} (\sphinxstyleliteralemphasis{float}) \textendash{} total discharge of the well

\item {} 
\sphinxstyleliteralstrong{rw} (\sphinxstyleliteralemphasis{float}) \textendash{} radius of the well

\item {} 
\sphinxstyleliteralstrong{res} (\sphinxstyleliteralemphasis{float}) \textendash{} resistance of the well screen

\item {} 
\sphinxstyleliteralstrong{layers} (\sphinxstyleliteralemphasis{int}\sphinxstyleliteralemphasis{, }\sphinxstyleliteralemphasis{array}\sphinxstyleliteralemphasis{ or }\sphinxstyleliteralemphasis{list}) \textendash{} layer (int) or layers (list or array) where well is screened

\item {} 
\sphinxstyleliteralstrong{label} (\sphinxstyleliteralemphasis{string}\sphinxstyleliteralemphasis{ or }\sphinxstyleliteralemphasis{None}\sphinxstyleliteralemphasis{ (}\sphinxstyleliteralemphasis{default: None}\sphinxstyleliteralemphasis{)}) \textendash{} label of the well

\end{itemize}

\end{description}\end{quote}
\paragraph{Examples}

\begin{sphinxVerbatim}[commandchars=\\\{\}]
\PYG{g+gp}{\PYGZgt{}\PYGZgt{}\PYGZgt{} }\PYG{n}{ml} \PYG{o}{=} \PYG{n}{Model3D}\PYG{p}{(}\PYG{n}{kaq}\PYG{o}{=}\PYG{l+m+mi}{10}\PYG{p}{,} \PYG{n}{z}\PYG{o}{=}\PYG{n}{np}\PYG{o}{.}\PYG{n}{arange}\PYG{p}{(}\PYG{l+m+mi}{20}\PYG{p}{,} \PYG{o}{\PYGZhy{}}\PYG{l+m+mi}{1}\PYG{p}{,} \PYG{o}{\PYGZhy{}}\PYG{l+m+mi}{2}\PYG{p}{)}\PYG{p}{,} \PYG{n}{kzoverkh}\PYG{o}{=}\PYG{l+m+mf}{0.1}\PYG{p}{)}
\PYG{g+gp}{\PYGZgt{}\PYGZgt{}\PYGZgt{} }\PYG{n}{Well}\PYG{p}{(}\PYG{n}{ml}\PYG{p}{,} \PYG{l+m+mi}{100}\PYG{p}{,} \PYG{l+m+mi}{200}\PYG{p}{,} \PYG{l+m+mi}{1000}\PYG{p}{,} \PYG{n}{layers}\PYG{o}{=}\PYG{p}{[}\PYG{l+m+mi}{0}\PYG{p}{,} \PYG{l+m+mi}{1}\PYG{p}{,} \PYG{l+m+mi}{2}\PYG{p}{,} \PYG{l+m+mi}{3}\PYG{p}{]}\PYG{p}{)}
\end{sphinxVerbatim}
\index{capzone() (timml.well.Well method)}

\begin{fulllineitems}
\phantomsection\label{\detokenize{wells/well:timml.well.Well.capzone}}\pysiglinewithargsret{\sphinxbfcode{capzone}}{\emph{nt=10}, \emph{zstart=None}, \emph{hstepmax=10}, \emph{vstepfrac=0.2}, \emph{tmax=None}, \emph{nstepmax=100}, \emph{silent=’.’}}{}
Compute a capture zone
\begin{quote}\begin{description}
\item[{Parameters}] \leavevmode\begin{itemize}
\item {} 
\sphinxstyleliteralstrong{nt} (\sphinxstyleliteralemphasis{int}) \textendash{} number of path lines

\item {} 
\sphinxstyleliteralstrong{zstart} (\sphinxstyleliteralemphasis{scalar}) \textendash{} starting elevation of the path lines

\item {} 
\sphinxstyleliteralstrong{hstepmax} (\sphinxstyleliteralemphasis{scalar}) \textendash{} maximum step in horizontal space

\item {} 
\sphinxstyleliteralstrong{vstepfrac} (\sphinxstyleliteralemphasis{float}) \textendash{} maximum fraction of aquifer layer thickness during one step

\item {} 
\sphinxstyleliteralstrong{tmax} (\sphinxstyleliteralemphasis{scalar}) \textendash{} maximum time

\item {} 
\sphinxstyleliteralstrong{nstepmax} (\sphinxstyleliteralemphasis{scalar}\sphinxstyleliteralemphasis{(}\sphinxstyleliteralemphasis{int}\sphinxstyleliteralemphasis{)}) \textendash{} maximum number of steps

\item {} 
\sphinxstyleliteralstrong{silent} (\sphinxstyleliteralemphasis{boolean}\sphinxstyleliteralemphasis{ or }\sphinxstyleliteralemphasis{string}) \textendash{} True (no messages), False (all messages), or ‘.’
(print dot for each path line)

\end{itemize}

\item[{Returns}] \leavevmode
\sphinxstylestrong{xyzt}

\item[{Return type}] \leavevmode
list of arrays of x, y, z, and t values

\end{description}\end{quote}

\end{fulllineitems}

\index{discharge() (timml.well.Well method)}

\begin{fulllineitems}
\phantomsection\label{\detokenize{wells/well:timml.well.Well.discharge}}\pysiglinewithargsret{\sphinxbfcode{discharge}}{}{}
The discharge in each layer
\begin{quote}\begin{description}
\item[{Returns}] \leavevmode
Discharge in each screen with zeros for layers that are not
screened

\item[{Return type}] \leavevmode
array (length number of layers)

\end{description}\end{quote}

\end{fulllineitems}

\index{headinside() (timml.well.Well method)}

\begin{fulllineitems}
\phantomsection\label{\detokenize{wells/well:timml.well.Well.headinside}}\pysiglinewithargsret{\sphinxbfcode{headinside}}{}{}
The head inside the well
\begin{quote}\begin{description}
\item[{Returns}] \leavevmode
Head inside the well for each screen

\item[{Return type}] \leavevmode
array (length number of screens)

\end{description}\end{quote}

\end{fulllineitems}

\index{plotcapzone() (timml.well.Well method)}

\begin{fulllineitems}
\phantomsection\label{\detokenize{wells/well:timml.well.Well.plotcapzone}}\pysiglinewithargsret{\sphinxbfcode{plotcapzone}}{\emph{nt=10, zstart=None, hstepmax=20, vstepfrac=0.2, tmax=365, nstepmax=100, silent=’.’, color=None, orientation=’hor’, win={[}-1e+30, 1e+30, -1e+30, 1e+30{]}, newfig=False, figsize=None}}{}
Plot a capture zone
\begin{quote}\begin{description}
\item[{Parameters}] \leavevmode\begin{itemize}
\item {} 
\sphinxstyleliteralstrong{nt} (\sphinxstyleliteralemphasis{int}) \textendash{} number of path lines

\item {} 
\sphinxstyleliteralstrong{zstart} (\sphinxstyleliteralemphasis{scalar}) \textendash{} starting elevation of the path lines

\item {} 
\sphinxstyleliteralstrong{hstepmax} (\sphinxstyleliteralemphasis{scalar}) \textendash{} maximum step in horizontal space

\item {} 
\sphinxstyleliteralstrong{vstepfrac} (\sphinxstyleliteralemphasis{float}) \textendash{} maximum fraction of aquifer layer thickness during one step

\item {} 
\sphinxstyleliteralstrong{tmax} (\sphinxstyleliteralemphasis{scalar}) \textendash{} maximum time

\item {} 
\sphinxstyleliteralstrong{nstepmax} (\sphinxstyleliteralemphasis{scalar}\sphinxstyleliteralemphasis{(}\sphinxstyleliteralemphasis{int}\sphinxstyleliteralemphasis{)}) \textendash{} maximum number of steps

\item {} 
\sphinxstyleliteralstrong{silent} (\sphinxstyleliteralemphasis{boolean}\sphinxstyleliteralemphasis{ or }\sphinxstyleliteralemphasis{string}) \textendash{} True (no messages), False (all messages), or ‘.’
(print dot for each path line)

\item {} 
\sphinxstyleliteralstrong{color} (\sphinxstyleliteralemphasis{color}) \textendash{} 

\item {} 
\sphinxstyleliteralstrong{orientation} (\sphinxstyleliteralemphasis{string}) \textendash{} ‘hor’ for horizontal, ‘ver’ for vertical, or ‘both’ for both

\item {} 
\sphinxstyleliteralstrong{win} (\sphinxstyleliteralemphasis{array\_like}\sphinxstyleliteralemphasis{ (}\sphinxstyleliteralemphasis{length 4}\sphinxstyleliteralemphasis{)}) \textendash{} {[}xmin, xmax, ymin, ymax{]}

\item {} 
\sphinxstyleliteralstrong{newfig} (\sphinxstyleliteralemphasis{boolean}\sphinxstyleliteralemphasis{ (}\sphinxstyleliteralemphasis{default False}\sphinxstyleliteralemphasis{)}) \textendash{} boolean indicating if new figure should be created

\item {} 
\sphinxstyleliteralstrong{figsize} (\sphinxstyleliteralemphasis{tuple of integers}\sphinxstyleliteralemphasis{, }\sphinxstyleliteralemphasis{optional}\sphinxstyleliteralemphasis{, }\sphinxstyleliteralemphasis{default: None}) \textendash{} width, height in inches.

\end{itemize}

\end{description}\end{quote}

\end{fulllineitems}


\end{fulllineitems}



\subsubsection{HeadWell}
\label{\detokenize{wells/headwell::doc}}\label{\detokenize{wells/headwell:headwell}}\index{HeadWell (class in timml.well)}

\begin{fulllineitems}
\phantomsection\label{\detokenize{wells/headwell:timml.well.HeadWell}}\pysiglinewithargsret{\sphinxstrong{class }\sphinxcode{timml.well.}\sphinxbfcode{HeadWell}}{\emph{model}, \emph{xw=0}, \emph{yw=0}, \emph{hw=10}, \emph{rw=0.1}, \emph{res=0}, \emph{layers=0}, \emph{label=None}}{}
HeadWell Class to create a well with a specified head inside the well.
The well may be screened in multiple layers. The resistance of the screen
may be specified. The head is computed such that the discharge \(Q_i\)
in layer \(i\) is computed as
\begin{equation*}
\begin{split}Q_i = 2\pi r_w(h_i - h_w)/c\end{split}
\end{equation*}
where \(c\) is the resistance of the well screen and \(h_w\) is
the head inside the well.
\begin{quote}\begin{description}
\item[{Parameters}] \leavevmode\begin{itemize}
\item {} 
\sphinxstyleliteralstrong{model} (\sphinxstyleliteralemphasis{Model object}) \textendash{} model to which the element is added

\item {} 
\sphinxstyleliteralstrong{xw} (\sphinxstyleliteralemphasis{float}) \textendash{} x-coordinate of the well

\item {} 
\sphinxstyleliteralstrong{yw} (\sphinxstyleliteralemphasis{float}) \textendash{} y-coordinate of the well

\item {} 
\sphinxstyleliteralstrong{hw} (\sphinxstyleliteralemphasis{float}) \textendash{} head inside the well

\item {} 
\sphinxstyleliteralstrong{rw} (\sphinxstyleliteralemphasis{float}) \textendash{} radius of the well

\item {} 
\sphinxstyleliteralstrong{res} (\sphinxstyleliteralemphasis{float}) \textendash{} resistance of the well screen

\item {} 
\sphinxstyleliteralstrong{layers} (\sphinxstyleliteralemphasis{int}\sphinxstyleliteralemphasis{, }\sphinxstyleliteralemphasis{array}\sphinxstyleliteralemphasis{ or }\sphinxstyleliteralemphasis{list}) \textendash{} layer (int) or layers (list or array) where well is screened

\item {} 
\sphinxstyleliteralstrong{label} (\sphinxstyleliteralemphasis{string}\sphinxstyleliteralemphasis{ (}\sphinxstyleliteralemphasis{default: None}\sphinxstyleliteralemphasis{)}) \textendash{} label of the well

\end{itemize}

\end{description}\end{quote}
\index{capzone() (timml.well.HeadWell method)}

\begin{fulllineitems}
\phantomsection\label{\detokenize{wells/headwell:timml.well.HeadWell.capzone}}\pysiglinewithargsret{\sphinxbfcode{capzone}}{\emph{nt=10}, \emph{zstart=None}, \emph{hstepmax=10}, \emph{vstepfrac=0.2}, \emph{tmax=None}, \emph{nstepmax=100}, \emph{silent=’.’}}{}
Compute a capture zone
\begin{quote}\begin{description}
\item[{Parameters}] \leavevmode\begin{itemize}
\item {} 
\sphinxstyleliteralstrong{nt} (\sphinxstyleliteralemphasis{int}) \textendash{} number of path lines

\item {} 
\sphinxstyleliteralstrong{zstart} (\sphinxstyleliteralemphasis{scalar}) \textendash{} starting elevation of the path lines

\item {} 
\sphinxstyleliteralstrong{hstepmax} (\sphinxstyleliteralemphasis{scalar}) \textendash{} maximum step in horizontal space

\item {} 
\sphinxstyleliteralstrong{vstepfrac} (\sphinxstyleliteralemphasis{float}) \textendash{} maximum fraction of aquifer layer thickness during one step

\item {} 
\sphinxstyleliteralstrong{tmax} (\sphinxstyleliteralemphasis{scalar}) \textendash{} maximum time

\item {} 
\sphinxstyleliteralstrong{nstepmax} (\sphinxstyleliteralemphasis{scalar}\sphinxstyleliteralemphasis{(}\sphinxstyleliteralemphasis{int}\sphinxstyleliteralemphasis{)}) \textendash{} maximum number of steps

\item {} 
\sphinxstyleliteralstrong{silent} (\sphinxstyleliteralemphasis{boolean}\sphinxstyleliteralemphasis{ or }\sphinxstyleliteralemphasis{string}) \textendash{} True (no messages), False (all messages), or ‘.’
(print dot for each path line)

\end{itemize}

\item[{Returns}] \leavevmode
\sphinxstylestrong{xyzt}

\item[{Return type}] \leavevmode
list of arrays of x, y, z, and t values

\end{description}\end{quote}

\end{fulllineitems}

\index{discharge() (timml.well.HeadWell method)}

\begin{fulllineitems}
\phantomsection\label{\detokenize{wells/headwell:timml.well.HeadWell.discharge}}\pysiglinewithargsret{\sphinxbfcode{discharge}}{}{}
The discharge in each layer
\begin{quote}\begin{description}
\item[{Returns}] \leavevmode
Discharge in each screen with zeros for layers that are not
screened

\item[{Return type}] \leavevmode
array (length number of layers)

\end{description}\end{quote}

\end{fulllineitems}

\index{headinside() (timml.well.HeadWell method)}

\begin{fulllineitems}
\phantomsection\label{\detokenize{wells/headwell:timml.well.HeadWell.headinside}}\pysiglinewithargsret{\sphinxbfcode{headinside}}{}{}
The head inside the well
\begin{quote}\begin{description}
\item[{Returns}] \leavevmode
Head inside the well for each screen

\item[{Return type}] \leavevmode
array (length number of screens)

\end{description}\end{quote}

\end{fulllineitems}

\index{plotcapzone() (timml.well.HeadWell method)}

\begin{fulllineitems}
\phantomsection\label{\detokenize{wells/headwell:timml.well.HeadWell.plotcapzone}}\pysiglinewithargsret{\sphinxbfcode{plotcapzone}}{\emph{nt=10, zstart=None, hstepmax=20, vstepfrac=0.2, tmax=365, nstepmax=100, silent=’.’, color=None, orientation=’hor’, win={[}-1e+30, 1e+30, -1e+30, 1e+30{]}, newfig=False, figsize=None}}{}
Plot a capture zone
\begin{quote}\begin{description}
\item[{Parameters}] \leavevmode\begin{itemize}
\item {} 
\sphinxstyleliteralstrong{nt} (\sphinxstyleliteralemphasis{int}) \textendash{} number of path lines

\item {} 
\sphinxstyleliteralstrong{zstart} (\sphinxstyleliteralemphasis{scalar}) \textendash{} starting elevation of the path lines

\item {} 
\sphinxstyleliteralstrong{hstepmax} (\sphinxstyleliteralemphasis{scalar}) \textendash{} maximum step in horizontal space

\item {} 
\sphinxstyleliteralstrong{vstepfrac} (\sphinxstyleliteralemphasis{float}) \textendash{} maximum fraction of aquifer layer thickness during one step

\item {} 
\sphinxstyleliteralstrong{tmax} (\sphinxstyleliteralemphasis{scalar}) \textendash{} maximum time

\item {} 
\sphinxstyleliteralstrong{nstepmax} (\sphinxstyleliteralemphasis{scalar}\sphinxstyleliteralemphasis{(}\sphinxstyleliteralemphasis{int}\sphinxstyleliteralemphasis{)}) \textendash{} maximum number of steps

\item {} 
\sphinxstyleliteralstrong{silent} (\sphinxstyleliteralemphasis{boolean}\sphinxstyleliteralemphasis{ or }\sphinxstyleliteralemphasis{string}) \textendash{} True (no messages), False (all messages), or ‘.’
(print dot for each path line)

\item {} 
\sphinxstyleliteralstrong{color} (\sphinxstyleliteralemphasis{color}) \textendash{} 

\item {} 
\sphinxstyleliteralstrong{orientation} (\sphinxstyleliteralemphasis{string}) \textendash{} ‘hor’ for horizontal, ‘ver’ for vertical, or ‘both’ for both

\item {} 
\sphinxstyleliteralstrong{win} (\sphinxstyleliteralemphasis{array\_like}\sphinxstyleliteralemphasis{ (}\sphinxstyleliteralemphasis{length 4}\sphinxstyleliteralemphasis{)}) \textendash{} {[}xmin, xmax, ymin, ymax{]}

\item {} 
\sphinxstyleliteralstrong{newfig} (\sphinxstyleliteralemphasis{boolean}\sphinxstyleliteralemphasis{ (}\sphinxstyleliteralemphasis{default False}\sphinxstyleliteralemphasis{)}) \textendash{} boolean indicating if new figure should be created

\item {} 
\sphinxstyleliteralstrong{figsize} (\sphinxstyleliteralemphasis{tuple of integers}\sphinxstyleliteralemphasis{, }\sphinxstyleliteralemphasis{optional}\sphinxstyleliteralemphasis{, }\sphinxstyleliteralemphasis{default: None}) \textendash{} width, height in inches.

\end{itemize}

\end{description}\end{quote}

\end{fulllineitems}


\end{fulllineitems}



\subsection{Line-sinks}
\label{\detokenize{linesinks/linesinkindex::doc}}\label{\detokenize{linesinks/linesinkindex:line-sinks}}
There are many line-sinks


\subsubsection{Head-specified line-sink}
\label{\detokenize{linesinks/headlinesink:head-specified-line-sink}}\label{\detokenize{linesinks/headlinesink::doc}}\index{HeadLineSink (class in timml.linesink)}

\begin{fulllineitems}
\phantomsection\label{\detokenize{linesinks/headlinesink:timml.linesink.HeadLineSink}}\pysiglinewithargsret{\sphinxstrong{class }\sphinxcode{timml.linesink.}\sphinxbfcode{HeadLineSink}}{\emph{model}, \emph{x1=-1}, \emph{y1=0}, \emph{x2=1}, \emph{y2=0}, \emph{hls=1.0}, \emph{res=0}, \emph{wh=1}, \emph{layers=0}, \emph{label=None}, \emph{addtomodel=True}}{}
\end{fulllineitems}



\subsubsection{HeadLineSinkString}
\label{\detokenize{linesinks/headlinesinkstring::doc}}\label{\detokenize{linesinks/headlinesinkstring:headlinesinkstring}}\index{HeadLineSinkString (class in timml.linesink)}

\begin{fulllineitems}
\phantomsection\label{\detokenize{linesinks/headlinesinkstring:timml.linesink.HeadLineSinkString}}\pysiglinewithargsret{\sphinxstrong{class }\sphinxcode{timml.linesink.}\sphinxbfcode{HeadLineSinkString}}{\emph{model, xy={[}(-1, 0), (1, 0){]}, hls=0, res=0, wh=1, order=0, layers=0, label=None, name=’HeadLineSinkString’}}{}
HeadLineSinkString Class to create a string of head-specified line-sinks
which may optionally have a width and resistance
:param model: Model to which the element is added
:type model: Model object
:param xy: elevation tops and bottoms of the aquifers from the top down
\begin{quote}

leaky layers may have zero thickness
if top=’conf’: length is 2 * number of aquifers
if top=’semi’: length is 2 * number of aquifers + 1 as top
of leaky layer on top of systems needs to be specified
\end{quote}
\begin{quote}\begin{description}
\item[{Parameters}] \leavevmode\begin{itemize}
\item {} 
\sphinxstyleliteralstrong{hls} (\sphinxstyleliteralemphasis{scalar}\sphinxstyleliteralemphasis{, }\sphinxstyleliteralemphasis{array}\sphinxstyleliteralemphasis{ or }\sphinxstyleliteralemphasis{list}) \textendash{} head along string
if scalar: head is the same everywhere along the string
if list or array of length 2: head at beginning and end of string
if list or array with same length as xy: heads at nodes, which
may contain nans, except for first and last point

\item {} 
\sphinxstyleliteralstrong{res} (\sphinxstyleliteralemphasis{scalar}\sphinxstyleliteralemphasis{ (}\sphinxstyleliteralemphasis{default is 0}\sphinxstyleliteralemphasis{)}) \textendash{} resistance of line-sink

\item {} 
\sphinxstyleliteralstrong{wh} (\sphinxstyleliteralemphasis{scalar}\sphinxstyleliteralemphasis{ or }\sphinxstyleliteralemphasis{str}) \textendash{} 

\item {} 
\sphinxstyleliteralstrong{order} (\sphinxstyleliteralemphasis{int}\sphinxstyleliteralemphasis{ (}\sphinxstyleliteralemphasis{default is 0}\sphinxstyleliteralemphasis{)}) \textendash{} order of all line-sinks in string

\item {} 
\sphinxstyleliteralstrong{layers} (\sphinxstyleliteralemphasis{scalar}\sphinxstyleliteralemphasis{, }\sphinxstyleliteralemphasis{list}\sphinxstyleliteralemphasis{ or }\sphinxstyleliteralemphasis{array}) \textendash{} layer(s) in which element is placed
if scalar: element is placed in this layer
if list or array: element is placed in all these layers

\item {} 
\sphinxstyleliteralstrong{label} (\sphinxstyleliteralemphasis{str}\sphinxstyleliteralemphasis{ or }\sphinxstyleliteralemphasis{None}) \textendash{} 

\item {} 
\sphinxstyleliteralstrong{-{-}-{-}-} \textendash{} 

\end{itemize}

\end{description}\end{quote}
\paragraph{Examples}

\begin{sphinxVerbatim}[commandchars=\\\{\}]
\PYG{g+gp}{\PYGZgt{}\PYGZgt{}\PYGZgt{} }\PYG{n}{ml} \PYG{o}{=} \PYG{n}{ModelMaq}\PYG{p}{(}\PYG{n}{kaq}\PYG{o}{=}\PYG{p}{[}\PYG{l+m+mi}{10}\PYG{p}{,} \PYG{l+m+mi}{20}\PYG{p}{]}\PYG{p}{,} \PYG{n}{z}\PYG{o}{=}\PYG{p}{[}\PYG{l+m+mi}{20}\PYG{p}{,} \PYG{l+m+mi}{12}\PYG{p}{,} \PYG{l+m+mi}{10}\PYG{p}{,} \PYG{l+m+mi}{0}\PYG{p}{]}\PYG{p}{,} \PYG{n}{c}\PYG{o}{=}\PYG{l+m+mi}{1000}\PYG{p}{)}
\PYG{g+gp}{\PYGZgt{}\PYGZgt{}\PYGZgt{} }\PYG{n}{HeadLineSinkString}\PYG{p}{(}\PYG{n}{ml}\PYG{p}{,} \PYG{n}{xy}\PYG{o}{=}\PYG{p}{[}\PYG{p}{(}\PYG{o}{\PYGZhy{}}\PYG{l+m+mi}{1}\PYG{p}{,} \PYG{l+m+mi}{0}\PYG{p}{)}\PYG{p}{,} \PYG{p}{(}\PYG{l+m+mi}{1}\PYG{p}{,} \PYG{l+m+mi}{0}\PYG{p}{)}\PYG{p}{,} \PYG{p}{(}\PYG{l+m+mi}{0}\PYG{p}{,} \PYG{l+m+mi}{1}\PYG{p}{)}\PYG{p}{]}\PYG{p}{,} \PYG{n}{hls}\PYG{o}{=}\PYG{l+m+mi}{5}\PYG{p}{)}
\end{sphinxVerbatim}

\end{fulllineitems}



\subsubsection{Line-sink ditch}
\label{\detokenize{linesinks/linesinkditch::doc}}\label{\detokenize{linesinks/linesinkditch:line-sink-ditch}}
Specified total discharge with unknown but uniform head
\index{LineSinkDitch (class in timml.linesink)}

\begin{fulllineitems}
\phantomsection\label{\detokenize{linesinks/linesinkditch:timml.linesink.LineSinkDitch}}\pysiglinewithargsret{\sphinxstrong{class }\sphinxcode{timml.linesink.}\sphinxbfcode{LineSinkDitch}}{\emph{model}, \emph{x1=-1}, \emph{y1=0}, \emph{x2=1}, \emph{y2=0}, \emph{Qls=1}, \emph{res=0}, \emph{wh=1}, \emph{order=0}, \emph{layers=0}, \emph{label=None}, \emph{addtomodel=True}}{}
\end{fulllineitems}



\subsection{Line-doublets}
\label{\detokenize{linedoublets/linedoubletindex:line-doublets}}\label{\detokenize{linedoublets/linedoubletindex::doc}}

\subsubsection{Impermeable wall}
\label{\detokenize{linedoublets/implinedoublet:impermeable-wall}}\label{\detokenize{linedoublets/implinedoublet::doc}}\index{ImpLineDoublet (class in timml.linedoublet)}

\begin{fulllineitems}
\phantomsection\label{\detokenize{linedoublets/implinedoublet:timml.linedoublet.ImpLineDoublet}}\pysiglinewithargsret{\sphinxstrong{class }\sphinxcode{timml.linedoublet.}\sphinxbfcode{ImpLineDoublet}}{\emph{model}, \emph{x1=-1}, \emph{y1=0}, \emph{x2=1}, \emph{y2=0}, \emph{order=0}, \emph{layers=0}, \emph{label=None}, \emph{addtomodel=True}}{}
\end{fulllineitems}



\subsubsection{Leaky wall}
\label{\detokenize{linedoublets/leakylinedoublet:leaky-wall}}\label{\detokenize{linedoublets/leakylinedoublet::doc}}\index{LeakyLineDoublet (class in timml.linedoublet)}

\begin{fulllineitems}
\phantomsection\label{\detokenize{linedoublets/leakylinedoublet:timml.linedoublet.LeakyLineDoublet}}\pysiglinewithargsret{\sphinxstrong{class }\sphinxcode{timml.linedoublet.}\sphinxbfcode{LeakyLineDoublet}}{\emph{model}, \emph{x1=-1}, \emph{y1=0}, \emph{x2=1}, \emph{y2=0}, \emph{res=0}, \emph{order=0}, \emph{layers=0}, \emph{label=None}, \emph{addtomodel=True}}{}
\end{fulllineitems}



\subsection{Uniform Flow}
\label{\detokenize{uflow::doc}}\label{\detokenize{uflow:uniform-flow}}\index{Uflow (class in timml.uflow)}

\begin{fulllineitems}
\phantomsection\label{\detokenize{uflow:timml.uflow.Uflow}}\pysiglinewithargsret{\sphinxstrong{class }\sphinxcode{timml.uflow.}\sphinxbfcode{Uflow}}{\emph{model}, \emph{slope}, \emph{angle}, \emph{name=’Uflow’}, \emph{label=None}}{}
\end{fulllineitems}



\subsection{Area-sinks}
\label{\detokenize{areasinks/areasinkindex::doc}}\label{\detokenize{areasinks/areasinkindex:area-sinks}}

\subsubsection{Circular Area-Sink}
\label{\detokenize{areasinks/circareasink::doc}}\label{\detokenize{areasinks/circareasink:circular-area-sink}}\index{CircAreaSink (class in timml.circareasink)}

\begin{fulllineitems}
\phantomsection\label{\detokenize{areasinks/circareasink:timml.circareasink.CircAreaSink}}\pysiglinewithargsret{\sphinxstrong{class }\sphinxcode{timml.circareasink.}\sphinxbfcode{CircAreaSink}}{\emph{model}, \emph{xc=0}, \emph{yc=0}, \emph{R=1}, \emph{N=0.001}, \emph{layer=0}, \emph{name=’CircAreasink’}, \emph{label=None}}{}
\end{fulllineitems}



\section{Utilities}
\label{\detokenize{utils/utils::doc}}\label{\detokenize{utils/utils:utilities}}


\renewcommand{\indexname}{Index}
\printindex
\end{document}